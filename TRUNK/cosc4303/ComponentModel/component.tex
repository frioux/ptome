\documentclass[12pt,titlepage]{article}
\usepackage{graphicx}
\usepackage{longtable}
\author{\textbf{The TOME Team}\\
Curtis ``Fjord'' Hawthorne \\
Craig Miller\\
Clint Olson\\
fREW Schmidt
}
\title{\textbf{TOME Component Model}}
\begin{document}
\maketitle
\tableofcontents
\listoffigures
\newpage
\section{Introduction}
\subsection{Purpose of Document}
The purpose of this document is to specify the TOME system and the component model it uses.

\subsection{Background}
In December of 2003, several students on Dorm 41 started a system called TOME.  The basic idea is that at the end of the semester, instead of everyone selling their books back to the bookstore, they all donate them to central repository.  Anyone on the floor can then check out whatever books they need free of charge for a semester.

The advantage of having a computer-based system to keep track of all those books is easy to see, and one has been under development ever since the start of TOME.  Since its humble beginnings as a quick solution over Christmas break, the system has grown to well over 3,000 lines of Perl code as well as HTML templates, a well-planned database schema, and significant documentation.  The system not only has comprehensive facilities for tracking books and patrons, but also keeps tabs on what books are used for what classes and other alternatives to purchasing new books.
\subsection{References}
\label{references}
All project data will be stored in a combination Subversion repository and Trac environment.  All of this will be made viewable at the following URL:

\texttt{http://enosh.letnet.net/trac/tome}
\end{document}

